\documentclass{article}
\usepackage{graphicx}
\usepackage[utf8]{inputenc}
\usepackage[fleqn]{amsmath}
\usepackage{titling}
\usepackage{graphicx,wrapfig,lipsum}
\usepackage{amssymb}
\usepackage{listings}
\usepackage[font=small,labelsep=none]{caption}

\setlength{\droptitle}{-10em}

\title{Project 2}\vspace{-3ex}
\author{Benedicte Allum Pedersen, Emil Heland Broll\\ Fredrik Oftedal Forr}
\date{\vspace{-5ex}}

\begin{document}
\maketitle

\section{Abstract}
	Eigenvalue problems, from the equations of a buckling beam to Schroedinger's equation for two electrons in a three-dimensional harmonic oscillator well

\section{Introduction}
	In this project we will solve eigenpair-problems using numerical calculations.
	We will also implement unit testing in our code to avoid mathematical and programming errors.

\section{Methods}
	\subsection{Mathematics}
		We will start off by proving that orhogonal or unitary transformations preserves the orthogonality/dot product of the vectors.

		Starting with an orthogonal basis of vectors, $\textbf{v}_i$, we know that:

		\begin{flalign*}
		\begin{aligned}
			\textbf{v}_i = \begin{bmatrix}
			v_{i1} \\
			\vdots \\
			v_{in}
			\end{bmatrix},
		\end{aligned}
		\qquad
		\begin{aligned}
			\textbf{v}_j^T \textbf{v}_i = \delta_{ij}
		\end{aligned}
		\end{flalign*}

		An orthogonal transformation gives us the following:
		\begin{flalign*}
		&\textbf{w}_i=U\textbf{v}_i, \text{where } U^TU=UU^T=\textbf{I}
		\end{flalign*}

		We want to prove that orthogonality is preserved:
		\begin{flalign*}
		\textbf{w}_i^T \textbf{w}_j = (U\textbf{v}_i)^T (U\textbf{v}_j) = \textbf{v}_i^T U^T U \textbf{v}_j = \textbf{v}_i^T \textbf{v}_j
		\end{flalign*}

		Now that we have proved that orthogonal transformations preserve orthogonality, we can move on to performing our Jacobi iterations. To do so, we use an orthogonal transformation matrix, S:
		\begin{flalign*}
			S = \begin{bmatrix}
				1 & 0 & \cdots \\
				0 & 1 &  0 \\
				\vdots & 0 & \ddots \\
				& & & \cos{\theta} & 0 & \cdots & \sin{\theta} \\
				& & & 0 & 1 & \cdots & 0 \\
				& & & \vdots & \vdots & \ddots & \vdots \\
				& & & -\sin{\theta} & 0 & \cdots & \cos{\theta}
			\end{bmatrix}, \qquad S^T=S^{-1}
		\end{flalign*}

		We simpltfy by assigning:
		\begin{flalign*}
			c = \cos{\theta}, s = \sin{\theta}, t = \tan{\theta} = \frac{s}{c}
		\end{flalign*}

		For our positive definite matrix ${A}$, we perform the transformation \\${S^TA S = B}$, giving us the general expression:
		\begin{flalign*}
		&	b_{ii} = a_{ii}, i \neq k, i \neq l \\
		&	b_{ik} = a_{ik}c - a_{il} s, i \neq k, i\neq l \\
		&	b_{il} = a_{il} c + a_{ik} s, i \neq k, i\neq l \\
		&	b_{kk} = a_{kk} c^2 - 2a_{kl}cs + a_{ll} s^2 \\
		&	b_{ll} = a_{ll} c^2 - 2a_{kl}cs + a_{kk} s^2 \\
		&	b_{kl} = (a_{kk}-a_{ll})cs + a_{kl}(c^2-s^2)
		\end{flalign*}

		By choosing the largest non-diagonal element in the original matrix $A$, we can fix $\theta$ by choosing to set the largest non-diagonal element to zero. From this, we can deduce the required values of $c$ and $s$:
		\begin{flalign*}
			c = \frac{1}{\sqrt(1+t^2)}, \qquad s = tc \\
			t = -\tau \pm \sqrt{1+\tau^2}, \qquad \tau = \frac{a_{ll}-a_{kk}}{2a_{kl}}
		\end{flalign*}

		This will result in a new matrix, B. We can then find the largest non-diagonal element of the new matrix and repeat the algorithm until this value is less than a given tolerance.
		\begin{flalign*}
			B_2 = S_2^T B S_2
		\end{flalign*}

		When this is achieved, we will have a diagonal matrix, where all non-diagonal matrix elements are approaching 0.
		\begin{flalign*}
			D = S_n^T S_{n-1}^T \cdots S_1^T A S_1 \cdots S_{n-1} S_n
		\end{flalign*}

		Since we have only performed orthogonal transformations, the eigenvalues for $D$ and $A$ will be the same. Since diagonal matrices have their eigenvalues on the diagonal, we can easily read off the eigenvalues.


	\subsection{Programming}
		To avoid errors in our code, we implement unit tests at the following points:\\
		* test
		* Test


\section{Results}
	For our eigenproblem-solver, we tested and found a reasonable tolerance, $10^{-8}$, so that the algorithm converges without reaching the maximum number of iterations, $n^3$. We test this using a tridiagonal matrix:
	\begin{flalign*}
		A =   \begin{bmatrix}
			2 & -1 & 0 &\dots & 0 & 0\\
			-1 & 2 & -1 & \dots & 0 & 0\\
			0 & -1 & 2 & \dots & 0 & 0 \\
			\vdots & \vdots & \vdots & \ddots & \vdots & \vdots \\
			0 & 0 & 0 &\dots& 2 & -1\\
			0 & 0 & 0 &\dots& -1 & 2
		\end{bmatrix}	
	\end{flalign*}

	Tabel 1 shows the number of similarity transformations were needed to reach a point where all non-diagonal matrix elements approach 0, for matrix dimensionalities.\\

	\begin{tabular}{c c c}
		Dimensionality & \# of transformations\\
		\hline
		4 & 45 \\
		10 & 1000 \textit{(wrong)} \\
		20 & 1092 \\
		50 & 433
	\end{tabular}

	We can see that it looks like \textit{dimensionality is proportional to the number of transformations (???)}.

	Table 2 shows a comparison of eigenvalues found with Armadillo's eigensystem-solver, on our tridiagonal matrix A.\\
	\begin{tabular}{c c c}
		$\lambda_{\text{Jacobi}}$ & $\lambda_{\text{Armadillo}}$ & Diff\\
		\hline
		0.0 & 0.0 & 0.0 \\
		FILL & ME & PLEASE
	\end{tabular} \\ \\

	Table 3 shows a comparison of runtimes of the Armadillo eigensystem-solver and our diagonalisation, for a 50x50-matrix.\\

	\begin{tabular}{c c c}
		Jacobi runtime & Armadillo runtime & Diff\\
		\hline
		0.0 & 0.0 & 0.0 \\
		FILL & ME & PLEASE
	\end{tabular}



\section{Conclusions}
	In this project we will solve eigenpair-problems using numerical calculations.





\end{document}
