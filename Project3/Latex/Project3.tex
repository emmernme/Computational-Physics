\documentclass{article}
\usepackage{graphicx}
\usepackage[utf8]{inputenc}
\usepackage[fleqn]{amsmath}
\usepackage{titling}
\usepackage{graphicx,wrapfig,lipsum}
\usepackage{amssymb}
\usepackage{listings}
\usepackage[font=small,labelsep=none]{caption}
\usepackage{hyperref}
\usepackage{caption}

\setlength{\droptitle}{-10em}
\setlength\parindent{0pt}

\title{Project 3}\vspace{-3ex}
\author{Benedicte Allum Pedersen, Emil Heland Broll\\ Fredrik Oftedal Forr}
\date{\vspace{-5ex}}

\begin{document}
\maketitle

\section*{Abstract}


\section*{Introduction}
In this report there will given a brief rundown on different numerical integration methods. The methods that have been used is Gauss-Legendre Quaderature, and Monte Carlo integration. There will be adjustments to these to show the different strengths of the methods. Under this paragraph you can find the method on how the program works, results from the program, discussion where we analyze the results and ending with a conclution. At the end there are an appendix where additional data is located.

\section*{Method}
In this problem there have been used c++ as programming language and mainly a variety of functions to solve the problem. 

\section*{Results}


\section*{Discussion}


\section*{Conclution}

3a)

N=15, $\lambda = 2$ gauss_legendre= 0.19947

N=31, $\lambda = 3.5$ gauss_legendre = 0.19397

N=35, $\lambda = 3.5$ gauss_legendre = 0.1914

N=27, $\lambda = 2.91$ gauss_legendre = 0.192801

N=27 $\lambda = 2.90$ gauss_legendre = 0.192725

---> dritbra

minker lambda så blir tilnermingen bedre


\section*{Appendix}



\end{document}
