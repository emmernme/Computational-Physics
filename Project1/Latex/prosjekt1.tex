\documentclass{article}
\usepackage{graphicx}
\usepackage[utf8]{inputenc}
\usepackage[fleqn]{amsmath}
\usepackage{titling}
\usepackage{graphicx,wrapfig,lipsum}
\usepackage{amssymb}
\usepackage{listings}
\usepackage[font=small,labelsep=none]{caption}

\setlength{\droptitle}{-10em}

\title{Project 1 - FYS3150}\vspace{-3ex}
\author{Emil Helland Broll, Benedicte Allum Pedersen,\\ Fredrik Oftedal Forr}
\date{\vspace{-5ex}}

\begin{document}
\maketitle

\section*{Project 1a)}

\begin{flalign*}
   &-\frac{u_{i+1}+u_{i-1}-2u_i}{h^2} = f_i\\
   &-(u_{i+1}+u_{i-1}-2u_i) = f_ih^2\\
   &2u_i -u_{i+1}-u_{i-1}=f_ih^2
\end{flalign*}
This expands to
\begin{flalign*}
  2u_1 - u_0 - u_2 &= f_1h^2\\
  2u_2 - u_1 - u_2 &= f_2h^2\\
  &\vdots\\
  2u_n - u_{n-1} - u_n+1 &= f_3h^2\\
\end{flalign*}
The boundary conditions give us $u_{n+1}=u(1)=0$ and $u_0=u(0)=0$. Then we can write this expression as
\begin{flalign*}
  \begin{bmatrix}
    2 & -1 & 0 &\dots & 0 & 0\\
    -1 & 2 & -1 & \dots & 0 & 0\\
    0 & -1 & 2 & \dots & 0 & 0 \\
    \vdots & \vdots & \vdots & \ddots & \vdots & \vdots \\
    0 & 0 & 0 &\dots& 2 & -1\\
    0 & 0 & 0 &\dots& -1 & 2
  \end{bmatrix}
  \begin{bmatrix}
    u_1\\
    u_2\\
    u_3\\
    \vdots\\
    u_{n-1}\\
    u_n
  \end{bmatrix} =
  \begin{bmatrix}
    f_1\\
    f_2\\
    f_3\\
    \vdots\\
    f_{n-1}\\
    f_n
  \end{bmatrix}
\end{flalign*}

\section*{Project 1b)}


\end{document}
