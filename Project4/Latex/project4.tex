\documentclass{article}
\usepackage{graphicx}
\usepackage[utf8]{inputenc}
\usepackage[fleqn]{amsmath}
\usepackage{titling}
\usepackage{graphicx,wrapfig,lipsum}
\usepackage{amssymb}
\usepackage{listings}
\usepackage[font=small,labelsep=none]{caption}

\setlength{\droptitle}{-10em}

\title{Project 4}\vspace{-3ex}
\author{Benedicte Allum Pedersen, Emil Heland Broll & Fredrik Oftedal Forr}
\date{\vspace{-5ex}}

\begin{document}
\maketitle

\section*{Abstract}
Si meg, hva betyr adjø?\\
Er det bare trist?\\
Noe som sårer deg?\\
Tro meg, vi skal ta adjø\\
Ikke sånn som sist\\
Da jeg gikk fra deg\\



\section*{Introduction}
In this project we will study the Ising model in two dimensions. This is a model which is used to simulate phase transitions. The model exhibits a phase transition from a magnetic phase to a phase with zero magnetization. The temperature where this phase transition occurs is called the critical temperature, $T_C$. Above this temperature the average magnetization is zero. We study electrons in a lattice which is a binary system because each electron only can take two values, spin up or spin down. \\

The energy we get from the Ising model without an externally applied magnetic field is given by:

\begin{flalign*}
  E = -J \sum_{<kl>}^N S_kS_l
\end{flalign*}

where $s_k, s_l = \pm 1$ and represents classical spin values. N is the total number of spins and J is a coupling constant expressing the strenght of the interactions between neighboring spins. $<kl>$ indicates that we sum over the spins of the nearest neighbors. We apply periodic boundry conditions as well as the Metropolis algorithm. We also assume that we have a ferromagnetic ordering, so $J > 0$.\\

The behavior og physical quantities like the mean magnetization, the heat capacity and the susceptibility can be characterized by a power law behavior when the temperature is near $T_C$. This gives:

\begin{flalign*}
  <M(T)> &\sim (T-T_C)^{\beta},\\
  C_v(T) &\sim |T_C-T|^{\alpha},\\
  \chi(T) &\sim |T_C-T|^{\gamma},
\end{flalign*}

where $\beta = 1/8, \alpha = 0$ and $\gamma = 7/4$. \\

The correlation length is another important physical quantity which can descriped like the ones above. The correlation length, $\varepsilon$, defines the length scale at which the overall properties of a material start to differ from its bulk properties(Jensen, M.). We expect $\varepsilon$ to be of the order of the lattice spacing for $T>>T_C$. As a result of more interactions between the spins as T approaches $T_C$ the correlation length increases as we get closer to $T_C$. Then the divergent behavior of $\varepsilon$ near $T_C$ is

\begin{flalign*}
  \varepsilon(T) \sim |T_C-T|^{-\nu}.
\end{flalign*}

We will always be limited to a finite lattice and $\varepsilon$ will be proportional with the size of the lattice. The behavior of a finite lattice can then be related to the behavior of a infinitely large lattice, so the critical temperature will scale as

\begin{flalign*}
  T_C(L) - T_C(L=\infty) = aL^{-1/\nu}.
\end{flalign*}

If we set $T=T_C$ the mean magnetization, the heat capacity and the susceptibility will be

\begin{flalign*}
  <M(T)> &\sim (T-T_C)^{\beta} \rightarrow L^{-\beta/\nu},\\
  C_v(T) &\sim |T_C-T|^{\alpha} \rightarrow L^{-\alpha/\nu},\\
  \chi(T) &\sim |T_C-T|^{\gamma} \rightarrow L^{-\gamma/\nu}.
\end{flalign*}


\section*{Method}
When calculating the degenerate energies for the case of 2x2, we start with the equation $E_i=-J\sum\limits_{\left<kl\right>}^{2}s_ks_l$\\
The case for all spin up looks like this
\begin{tabular}{c c}
  $\uparrow$ & $\uparrow$\\
  $\uparrow$ & $\uparrow$
\end{tabular}\\

And the equation will be.
\begin{flalign*}
  E_1 &= -J\sum\limits_{<kl>}^{2}s_k s_l\\
  &= -J((s_1 s_2+s_1 s_3)+(s_2 s_1+s_2s_4)+(s_3s_1+s_3s_4)+(s_4s_3+s_4s_2))\\
  &= -J((1+1) + (1+1) + (1+1) + (1+1))\\
  E_1 &= -8J
\end{flalign*}
The reason why the same interaction is included several times is because of the unit cell repeating itself to inflinity in both x and y derection. Therefore the $s_1$ will interact with $s_2$ and $s_3$ inside the unit cell, and $s_2$ and $s_3$ "outside" the unit cell.

When this is known for all the degenerate energies we can calculate the walue of the partian function.
\begin{flalign*}
  &z = \sum\limits_{i=1}^{2^n}e^{-\beta E_i}\\
  &\text{In out case we have $n=4$ since we have a 2x2 lattice.}\\
  &z = \sum\limits_{i=1}^{2^4}e^{-\beta E_i}\\
  &z = e^{-\beta E_1}+e^{-\beta E_2}+\hdots+e^{-\beta E_16}\\
  &z = e^{8 \beta J}+4e^{- \beta \cdot 0} + 2e^{-8 \beta J} + 4e^{-\beta \cdot 0} + 4e^{-\beta \cdot 0} + 4e^{-\beta \cdot 0} + e^{8 \beta J}\\
  &z = 2e^{8 \beta J} + 2e^{-8 \beta J} + 16
\end{flalign*}


This gives us the ability to calculate the expectationvalue of the energy $\left<E\right>$

\begin{flalign*}
  \left<E\right> &= \sum\limits_{i}^{2^n}\frac{E_i e^{-\beta E_i}}{z}\\
  &\text{We know we have several energyvalues which is zero. If we do not write these we get}\\
  &= \frac{-8Je^{8\beta J} +2\left(8Je^{-8 \beta J} \right) + \left(-8J\right)e^{8\beta J}} {2e^{8 \beta J} + 2e^{-8 \beta J} + 16}\\
  &= \frac{16J\left(e^{-8\beta J}- e^{8 \beta J} \right) }{2 {\left(e^{8 \beta J} + e^{-8 \beta J} + 8 \right)} }\\
  &= 8J \frac{e^{-8\beta J}- e^{8 \beta J}}{e^{8 \beta J} + e^{-8 \beta J} + 8}
\end{flalign*}

Thus
\begin{flalign*}
  \left<E\right>^2 &= \left( 8J \frac{e^{-8\beta J}- e^{8 \beta J}}{e^{8 \beta J} + e^{-8 \beta J} + 8} \right)^2\\
  &=
\end{flalign*}

At the same time we calculate $\left<E^2\right>$

\begin{flalign*}
  \left<E^2\right> &= \sum\limits_{i}^{2^n}\frac{E_i^2 e^{-\beta E_i}}{z}\\
  &= \frac{64Je^{8\beta J} + 2\left(64e^{-8\beta J}\right) + 64Je^{8\beta J}}{2e^{8 \beta J} + 2e^{-8 \beta J} + 16}\\
  &= \frac{128Je^{8\beta J} + 128e^{-8\beta J}}{2e^{8 \beta J} + 2e^{-8 \beta J} + 16}\\
  &= \frac{64Je^{8\beta J} + 64e^{-8\beta J}}{e^{8 \beta J} + e^{-8 \beta J} + 8}
\end{flalign*}



When this is known we can also calculate the heat capasitace.


\end{document}
