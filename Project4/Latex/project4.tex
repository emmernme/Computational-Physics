\documentclass{article}
\usepackage{graphicx}
\usepackage[utf8]{inputenc}
\usepackage[fleqn]{amsmath}
\usepackage{titling}
\usepackage{graphicx,wrapfig,lipsum}
\usepackage{amssymb}
\usepackage{listings}
\usepackage[font=small,labelsep=none]{caption}
\usepackage{array}% http://ctan.org/pkg/array
\usepackage{lipsum}
\usepackage{subcaption}
\usepackage{float}


% Ignorer teite advarsler i LaTeX
\pretolerance=10000
\tolerance=10000
\overfullrule=1mm
\sloppy
\hbadness=10000


\setlength{\droptitle}{-10em}

\title{Project 4}\vspace{-3ex}
\author{Benedicte Allum Pedersen, Emil Helland Broll and Fredrik Oftedal Forr}
\date{\vspace{-5ex}}

\begin{document}
\maketitle

\section{Abstract}
	The Ising model in two dimensions has been used to simulate phase transitions in a magnetic system. We have used Monte Carlo and the Metropolis algorithm and our results showes that the number of Monte Carlo cycles wee need to reach equilibrium depends on the spin configuration in the start matrix and also on the temperature. We have estimated the

\newpage

\tableofcontents{}

\newpage

\section{Introduction}
	In this project we will study the Ising model in two dimensions. This model is used to simulate phase transitions in materials. A magnetic material will exhibit a phase transition from a magnetic phase to a phase with zero magnetization when the temperature of the system increases. The temperature where this phase transition occurs is called the critical temperature, $T_C$, where at temperatures above $T_C$, the average magnetization is zero. We will be studying the spins of electrons in a lattice, which is a binary system because each electron only can take two values, spin up or spin down, which we represent by +1 and -1. \\

    \subcaption{Energy of the System}

	The energy we get from the Ising model without an externally applied magnetic field is given by:

	\begin{flalign*}
		E = -J \sum^N_{<kl>} S_k S_l
	\end{flalign*}

	where $s_k, s_l = \pm 1$ and represents classical spin values of electrons at position $k$ and position $l$ in a grid. N is the total number of spins (electrons) and J is a coupling constant expressing the strenght of the interactions between neighboring spins. $<kl>$ indicates that we sum over the spins of the nearest neighbors.

	In the model, we apply periodic boundary conditions, meaning that for electrons on the edge of the 2D grid, their nearest neighbors on the edge side is the electron by the opposite edge. In addition, we assume that we have a ferromagnetic ordering, meaning that $J > 0$.

    \subsection{Phase Transitions}

	The behavior of physical quantities like the mean magnetization, the heat capacity and the susceptibility can be characterized by a power law behavior when the temperature is near $T_C$:

	\begin{flalign*}
		\langleM(T)\rangle &\sim (T-T_C)^{\beta},\\
		C_v(T) &\sim |T_C-T|^{\alpha},\\
		\chi(T) &\sim |T_C-T|^{\gamma},
	\end{flalign*}

	where $\beta = 1/8, \alpha = 0$ and $\gamma = 7/4$. \\

	The correlation length is another important physical quantity which can be described in the same way. The correlation length, $\varepsilon$, defines the length scale at which the overall properties of a material start to differ from its bulk properties (Jensen, M.). We expect $\varepsilon$ to be of the order of the lattice spacing for $T>>T_C$. As a result of more interactions between the spins as T approaches $T_C$, the correlation length increases as we get closer to $T_C$. This means that the divergent behavior of $\varepsilon$ near $T_C$ is:

	\begin{flalign*}
		\varepsilon(T) \sim |T_C-T|^{-\nu}.
	\end{flalign*}

	We will always be limited to a finite lattice and $\varepsilon$ will be proportional with the size of the lattice. The behavior of a finite lattice can then be related to the behavior of a infinitely large lattice, so the critical temperature will scale as

	\begin{flalign*}
		T_C(L) - T_C(L=\infty) = aL^{-1/\nu}.
	\end{flalign*}

	If we set $T=T_C$, the mean magnetization, the heat capacity and the susceptibility will be

	\begin{flalign*}
		\left< M(T)\right> &\sim (T-T_C)^{\beta} \rightarrow L^{-\beta/\nu},\\
		C_v(T) &\sim |T_C-T|^{\alpha} \rightarrow L^{-\alpha/\nu},\\
		\chi(T) &\sim |T_C-T|^{\gamma} \rightarrow L^{-\gamma/\nu}.
	\end{flalign*}

	In this project, we will use a discretised model of the system and run numerical processes to calculate the physical quantities of the system in order to study the phase transitions in the material.

\section{Method}
	In the Appendix, we have described how we calcuate the degenerated energies and magnetization of a 2x2-grid of spins. Table \ref{Tab: EogM} shows the number of possible configurations that result in the same energy and magnetization, for a given number of spins pointing up. In total, 16 different spin configurations are possible for a 2x2 grid but only 3 different energies and 3 different magnetizations.

	\begin{table}[h!]
		\caption{: Spinconfigurations grouped by their total energy and magnetization}
			\label{Tab: EogM}
			\centering
		\begin{tabular}{c c c c}
			$\#$ spins up & $\#$ configurations & $E^2$ & M \\
			\hline
			4 & 1 & -8J & 4 \\
			3 & 4 & 0 & 2 \\
			2 & 4 & 0 & 0 \\
			2 & 2 & +8J & 0\\
			1 & 4 & 0 & -2 \\
			0 & 1 & -8J & -4 \\
		\end{tabular}
	\end{table}

	We have used the values in Table \ref{Tab: EogM} to calculate the expectation values for the energy and the mean magnetization. These values have then been used to calculate the variance for the two physical quantities. The variance of the energy and the mean magnetization have respectively been used to calculate the heat capacity $c_v &= \sigma^2_E/k_BT^2$ and the susceptibility $\chi = \sigma_M^2/k_BT$

	\subsection{The Metropolis Algorithm}
		The Metropolis algorithm uses a propability distribution to obtain a sequence of random samples. In our case it is used to decide if the spins should flip or not. If we have a uniform propability distribution function(PDF), $\omega \in [0,1]$ this is given by:

        \begin{flalign*}
            \omega &= \frac{P_i}{P_j} = \frac{P_{new\:state}}{P_{previous\: state}}
             = \frac{e^{-\beta E_i/z}}{e^{-\beta E_j /z}}\\
             \qquad\\
            \omega &= e^{-\beta(E_i-E_j)} = e^{-\beta \Delta E}\\
        \end{flalign*}

        \noindent where $\Delta E $ is the energy difference between the new and the previous state, $E_i$ and $E_j$. $\beta = 1/k_BT$ and Z is the partition function described in the Appendix(Section 7.2).
        If $\omega = 1$ the system will stay at the same place, there will be no change in the spin configuration. If $\omega > 1$ the system will move to a larger propability of spin configurations and if $\omega < 1$ the system move to a smaller propability.

	\subsection*{Programming}
		In order to calculate the spin energies for different systems, we implement the Ising model using the Monte Carlo method. This is a mostly random method, where we perform a loop over random positions in the system grid of spins and decide whether or not we should flip spins using the Metropolis algorithm. \\

		The most effective way to run this simulation is to loop over one random position at a time and decide if we should flip that one spin or not. Because we can easily calculate the local energy contribution of that one spin, if we keep track of the current enery of the system before flipping, it is easy to calculate the new total energy, so we don't need to loop over the full grid each time we want to find the total energy. \\

		In the Ising model, we know that when $N \rightarrow \infty$, the system should reach equilibrium. For our model, N is the number of Monte Carlo cycles, so we want to keep N as large as possible while still being able to run the calculations in a reasonable amount of time. This means we need to optimize our Monte Carlo loop as much as possible.\\

		In theory, what we need to to within the MC loop is the following:
		\begin{itemize}
			\item Find the spin at a random position
			\item Calculate the current local energy contribution of that spin
			\item Calculate the new total energy of the system if the spin is flipped
			\item Use the Metropolis algorithm to decide if we should flip or not
			\item If we flip: calculate the new total energy and new total magnetization
			\item Update the expectation values with the (new) energy and total magnetization
		\end{itemize}

		Because we are looking at a 2D system, the energies actually get surprisingly easy to calculate. Because we know we only flip one spin at a time, the surrounding four spins (above, below, left and right) stay the same. The consequence of this is that the local energy contribution can only ever "flip" – if you flip the centre spin, the local energy will be $E_{new}(x,y) = -E_{old}(x,y))$. Since the local energy only ever depends on five spins, the possible local energies are also easily calculated. The only possible energies of a local part of the system are: $E(x,y) = {-4J, -2J, 0, 2J, 4J}$, making the only possible transition energies: $\Delta E(x,y) = {8J, 4J, 0, -4J, -8J}$

		We can use this knowledge to pre-calculate the probabilities of accepting a flip from one energy to another, using the Metropolis algorithm. Our requirement for flipping is:
		\begin{equation}
			r \leq e^{-\beta \Delta E}
		\end{equation}

		This, again, can be simplified to say that as long as $\Delta E \leq 0$, accept the flip, to avoid having to deal with the random number at all, saving some time.

		These simplifications should make our code run quite fast, in theory, fast enough that we don't need to parallelize the Monte Carlo loop itself. However, if we want to run multiple Monte Carlo loops of millions or billions of cycles, we should definitely parallelize in order to utilze our full computing power.

		In this project, we parallelize using Open MPI, dividing the different Monte Carlo loops among the processors we have available, and feed the results back to the main process for analysis.

\section{Results}
	In order to test our algorithm and its precision, we compare the analytical values of the expectation values of the energy and magnetization to our program's equivalent values. We ran the test by starting with a low number of MC cycles and increasing the number gradually until we got a good agreement.
	For a 2x2 lattice, we approximately need 10 000 Monte Carlo cycles in order to achieve a good agreement with the analytical values of the system. Using 10 000 MC cycles, we get $\left<E\right> = -7.9864 J$, which is only $0.007814 J$ from the analytical value. The values described in the sections above are represented in Table \ref{Tab: values}, where we have set $J/k_BT =\beta J = 1$. The calculations for the analytical values can be found in the Appendix.

	{\renewcommand{\arraystretch}{1.5}
	\begin{table}[h!]
		\caption{: Analytical and numerical values for T = 1.0, L = 2 and $10^7$ Monte Carlo cycles }
			\label{Tab: values}
			\centering
		\begin{tabular}{c c c c}
			Property & Analytical & Analytical per spin & Numerical per spin \\
			\hline
			$\left<E\right>$ & -7.978586  & -1.994647 & -1.99594\\
			$\left<E^2\right>$ & 63.65786 & 15.91447 &15.9675 \\
			$C_v$ & 0.170865 & 0.042716 & 0.0324227 \\
			$\left<M(T)\right>$  & 0 & 0 & -0.0025167\\
			$\left<|M(T)|\right>$ & 3.991970 & 0.997993 & 0.998647 \\
			$\left<|M(T)|^2\right> $ & 15.962527 & 3.990631 & 3.99323\\
			$\chi$ & 0.026703 & 0.006676 & 0.00202647\\
		\end{tabular}
	\end{table}

\subsection{Number of Monte Carlo Cycles and Equilibration Times}
	In order to find a more precise measurement of the number of MC cycles needed to reach equilibrium, we run a test with a 20x20 lattice. The test involves logging the mean energy and mean magnetization for each Monte Carlo cycle, for both $T=1.0$ and $T=2.3$. Figure \ref{fig:steady_E} shows the results for two different initial spin matrices, one with all spins up and one random matrix.

	For a matrix with a random spin configuration at $T=1.0$, we reach steady state at around 200 000 Monte Carlo cycles, while for the matrix with all spins up we seemingly only need a few cycles.

	\begin{figure}[H]
	\makebox[\textwidth]{\makebox[1.5\textwidth]{%
	\begin{subfigure}{.5\textwidth}
			\centering
			\includegraphics[width=280px]{./plot/L20_1mill.png}
			\caption{Initial matrix with all spins up. }
	\end{subfigure} \hfill %
	\begin{subfigure}{.5\textwidth}
			\centering
			\includegraphics[width=280px]{./plot/random_L20_1mill.png}
			\caption{Initial matrix with random spins.}
	\end{subfigure}\hfill}}
	\caption{: Energy and magnetization plotted against number of Monte Carlo cycles when $T = 1.0$.}
	\label{fig:steady_E}
	\end{figure}

	When we increase the temperature, we see that we need substantially more Monte Carlo cycles to achive a steady state. Figure \ref{fig:steady_E_highT} shows that we would need around 1 million Monte Carlo cycles for both the random matrix and the matrix with all spins up. An estimate for the actual time it takes to reach equilibrium based on the number of Monte Carlo cycles can be made if we assume that one Monte Carlo cycle equals 1 second. The estimated equilibration times can be found in Table \ref{Tab:equilibration_times}.

	\begin{figure}[H]
	\makebox[\textwidth]{\makebox[1.5\textwidth]{%
	\begin{subfigure}{.5\textwidth}
			\centering
			\includegraphics[width=280px]{./plot/L20_5mill_highT.png}
			\caption{Initial matrix with all spins up. }
	\end{subfigure} \hfill %
	\begin{subfigure}{.5\textwidth}
			\centering
			\includegraphics[width=280px]{./plot/random_L20_5mill_highT.png}
			\caption{Initial matrix with random spins.}
	\end{subfigure}\hfill}}
	\caption{: Energy and magnetization plotted against number of Monte Carlo cycles when $T = 2.4$. }
	\label{fig:steady_E_highT}
	\end{figure}

	{\renewcommand{\arraystretch}{1.5}
	\begin{table}[h!]
		\caption{: Estimated equilibration times.}
			\label{Tab:equilibration_times}
			\centering
		\begin{tabular}{c c}
				Temperature & Matrix & Equilibration time (sec)\\
				\hline
				$1.0$ & Random spins & 200 000  \\
				$1.0$ & All spins up & $\approx$ 0 \\
				$2.4$ & Random spins & 1 000 000\\
				$2.4$ & All spins up & 1 000 000\\
			\hline
		\end{tabular}
	\end{table}

	Figure \ref{fig:flips} and Figure \ref{flips_random} show plots of the total number of accepted configurations as function of the total number of Monte Carlo cycles for respectively an initial matrix with all spins up and an initial matrix with random spins.

	\begin{figure}[H]
	\makebox[\textwidth]{\makebox[1.5\textwidth]{%
	\begin{subfigure}{.5\textwidth}
			\centering
			\includegraphics[width=280px]{./plot/number_of_flips.png}
			\caption{T = 1.0}
	\end{subfigure} \hfill %
	\begin{subfigure}{.5\textwidth}
			\centering
			\includegraphics[width=280px]{./plot/number_of_flips_highT.png}
			\caption{T = 2.4}
	\end{subfigure}\hfill}}
	\caption{: Total number of accepted spin configuration as a function of Monte Carlo cycles, for an initial matrix with all spins up. }
	\label{fig:flips}
	\end{figure}

	\begin{figure}[H]
	\makebox[\textwidth]{\makebox[1.5\textwidth]{%
	\begin{subfigure}{.5\textwidth}
			\centering
			\includegraphics[width=280px]{./plot/random_number_of_flips.png}
			\caption{T = 1.0}
	\end{subfigure} \hfill %
	\begin{subfigure}{.5\textwidth}
			\centering
			\includegraphics[width=280px]{./plot/random_number_of_flips_highT.png}
			\caption{T = 2.4}
	\end{subfigure}\hfill}}
	\caption{: Total number of accepted spin configuration as a function of Monte Carlo cycles, for an initial matrix with all spins up. }
	\label{fig:flips_random}
	\end{figure}

\subsection{Propability Distribution}
	% d)
	The probability distribution of the energy for $T = 1.0$ and $T = 2.4$ is shown in figure \ref{fig:probability}

	\begin{figure}[H]
	\makebox[\textwidth]{\makebox[1.5\textwidth]{%
	\begin{subfigure}{.5\textwidth}
			\centering
			\includegraphics[width=280px]{./plot/histogram_random_lowT.png}
			\caption{$T = 1.0$}
	\end{subfigure} \hfill %
	\begin{subfigure}{.5\textwidth}
			\centering
			\includegraphics[width=300px]{./plot/histogram_random_highT.png}
			\caption{$T = 2.4$}
	\end{subfigure}\hfill}}
	\caption{: Probability distribution for counted energies.}
	\label{fig:probability}
	\end{figure}

\subsection{Phase Transitions}


\section{Discussion}

	%b) Compare your results with the expressions from a) for a temperature T = 1.0 (in units of kT /J).
	%How many Monte Carlo cycles do you need in order to achieve a good
	%agreeement?

	The numerical value for $<E>, <E^2>$ and $C_v$ for a 2x2 lattice is in good agreement with the analytical values. However, the numerical values for the magnetization and the susceptibility does not.

	The mean magnetization varies a lot in different calculations, because the magnetization is heavily dependent on each single spin flip. While the energy dependence on each flip is the square value of the product of the neighboring spins, the magnetization depends on the sum of all spins. This might be why we experience different values for the mean magnetization in different calculations.

	% c)
	From Figure \ref{fig:steady_E} and \ref{fig:steady_E_random} we see that the energy and magnetization needs a lot less time to stabelize when all of the spins in the lattice have the same configuration, when $T = 1.0$. When we increase the temperature, the number of Monte Carlo cycles we need to get good results also increase. For a higher temperature we can also see that the energy in the system increases. As mentioned above, the magnetization is more dependent on the spin configuration in the entire matrix and therefore it will not be as stable as the energy.


	%Vi må kommentere Figure \ref{fig:flips} and Figure %\ref{flips_random}
	We can see that the variance increases with temperature, explaining why we get a more spread out probability distribution of the energy when the temperature increases.

\section{Conclusion}


\section{Appendix}
	\subsection{Degenerated Energies and Magnetization}

		When calculating the degenerate energies for a 2x2 lattice, we start with the equation $E_i=-J\sum\limits_{\left<kl\right>}^{2}s_ks_l$\\
		The case for all spins up looks like this:
		\begin{tabular}{c c}
			$\uparrow$ & $\uparrow$\\
			$\uparrow$ & $\uparrow$
		\end{tabular}\\

		The energy for this system is:
		\begin{flalign*}
			E_1 &= -J\sum\limits_{<kl>}^{2}s_k s_l\\
			&= -J((s_1 s_2+s_1 s_3)+(s_2 s_1+s_2s_4)+(s_3s_1+s_3s_4)+(s_4s_3+s_4s_2))\\
			&= -J((1+1) + (1+1) + (1+1) + (1+1))\\
			E_1 &= -8J
		\end{flalign*}
		In this calculation, we can see that the same interactions are counted twice. The reason for this is that the unit cell repeats itself to infinity in both x and y direction, which is our periodic boundary condition. Therefore the spin $s_1$ will interact with $s_2$ and $s_3$ inside of our unit cell, while $s_2$ and $s_3$ interact "outside" of the unit cell.

		The magnetization is defined as:
		\begin{flalign*}
			M_i=\sum_{j=1}^{N} s_j
		\end{flalign*}
		which sums over all spins for a given configuration $i$. For the same spin configuration as above, this gives us:
		\begin{flalign*}
			M_i=\sum_{j=1}^{4} (1 + 1 + 1 +1) = 4
		\end{flalign*}

	\subsection{The Partition Function}
		When we known the values for all the degenerate energies we can calculate the value of the partition function.
		\begin{flalign*}
			&z = \sum\limits_{i=1}^{2^n}e^{-\beta E_i}\\
			&\text{For a 2x2-lattice, $n=4$.}\\
			&z = \sum\limits_{i=1}^{2^4}e^{-\beta E_i}\\
			&z = e^{-\beta E_1}+e^{-\beta E_2}+\hdots+e^{-\beta E_16}\\
			&z = e^{8 \beta J}+4e^{- \beta \cdot 0} + 2e^{-8 \beta J} + 4e^{-\beta \cdot 0} + 4e^{-\beta \cdot 0} + 4e^{-\beta \cdot 0} + e^{8 \beta J}\\
			&z = 2e^{8 \beta J} + 2e^{-8 \beta J} + 16
		\end{flalign*}

	\subsection{Expectation Values for the Energy}
		The partition function is used to calculate the expectation value of the energy, $\left<E\right>$.

		\begin{flalign*}
			\left<E\right> &= \sum\limits_{i}^{2^n}\frac{E_i e^{-\beta E_i}}{z}\\
			\textit{Terms involving energies that are 0 are omitted below.}\\
			&= \frac{-8Je^{8\beta J} +2\left(8Je^{-8 \beta J} \right) + \left(-8J\right)e^{8\beta J}} {2e^{8 \beta J} + 2e^{-8 \beta J} + 16}\\
			&= \frac{16J\left(e^{-8\beta J}- e^{8 \beta J} \right) }{2 {\left(e^{8 \beta J} + e^{-8 \beta J} + 8 \right)} }\\
			&= 8J \frac{e^{-8\beta J}- e^{8 \beta J}}{e^{8 \beta J} + e^{-8 \beta J} + 8}
		\end{flalign*}

		Thus:
		\begin{flalign*}
			\left<E\right>^2 &= \left( 8J \frac{e^{-8\beta J}- e^{8 \beta J}}{e^{8 \beta J} + e^{-8 \beta J} + 8} \right)^2\\
			&= 64J^2 \frac{\left(e^{-8\beta J} - e^{8\beta J} \right)^2 }{\left(e^{8\beta J} + e^{-8 \beta J} + 8 \right)^2 }\\
			&= 64 J^2 \frac{e^{-16\beta J} + e^{16 \beta J} + 2}{e^{16\beta J} + e^{-16\beta J} + 16e^{-8\beta J} + 16e^{8 \beta J} + 66 }
		\end{flalign*}

		In order to calculate the variance, we need the expectation value of $E^2$.

		\begin{flalign*}
			\left<E^2\right> &= \sum\limits_{i}^{2^n}\frac{E_i^2 e^{-\beta E_i}}{z}\\
			&= \frac{64J^2e^{8\beta J} + 2\left(64J^2e^{-8\beta J}\right) + 64J^2e^{8\beta J}}{2e^{8 \beta J} + 2e^{-8 \beta J} + 16}\\
			&= \frac{128J^2e^{8\beta J} + 128J^2e^{-8\beta J}}{2e^{8 \beta J} + 2e^{-8 \beta J} + 16}\\
			&= 64J^2\frac{e^{8\beta J} + e^{-8\beta J}}{e^{8 \beta J} + e^{-8 \beta J} + 8}
		\end{flalign*}

	\subsection{Expectation Values for the Magnetization}
		The absolute value of the mean magnetization is given by
		\begin{flalign*}
			\left<|M(T)|\right> = \frac{\sum \limits{_i^{2^4}} |M_i| e^{-\beta E_i}}{z}
		\end{flalign*}
		and can be calculated by using the values for the energy and the magnetization from Table \ref{Tab: EogM}:

		\begin{flalign*}
			\left<|M(T)|\right> &= \frac{4e^{8\beta J} + 2\cdot4 + 0\cdot4 + 0\cdot2 + |-2|\cdot4 + |-4|e^{8\beta J}}{z}\\
			\left<|M(T)|\right> &= \frac{8e^{8\beta J} + 16}{2e^{8 \beta J} + 2e^{-8 \beta J} + 16} = \frac{4e^{8\beta J} + 8}{e^{8 \beta J} + e^{-8 \beta J} + 8}
		\end{flalign*}

		We also have that
		\begin{flalign*}
			\left<|M(T)|^2\right> &=  \frac{\sum \limits{_i^{2^4}} |M_i^2| e^{-\beta E_i}}{z}\\
			\left<|M(T)|^2\right> &= \frac{32e^{8\beta J} + 32}{2e^{8 \beta J} + 2e^{-8 \beta J} + 16} = 16\frac{e^{8\beta J} + 1}{e^{8 \beta J} + e^{-8 \beta J} + 8}
		\end{flalign*}

	\subsection{Heat Capacity}
		We use the expectation values of the energy to calculate the variance:
		\begin{flalign*}
			\sigma^2_E &= \left<E^2\right> - \left<E\right>^2\\
			&= 64J^2\frac{e^{8\beta J} + e^{-8\beta J}}{e^{8 \beta J} + e^{-8 \beta J} + 8} - 64 J^2 \frac{e^{-16\beta J} + e^{16 \beta J} + 2}{e^{16\beta J} + e^{-16\beta J} + 16e^{-8\beta J} + 16e^{8 \beta J} + 66 }
		\end{flalign*}

		The heat capacity is defined as the variance divided by the energy and temperature squared.
		\begin{flalign*}
			C_v &= \frac{\sigma^2_E}{k_BT^2}\\
			&= \frac{64J^2}{k_BT^2}\left[\frac{e^{8\beta J} + e^{-8\beta J}}{e^{8 \beta J} + e^{-8 \beta J} + 8} -  \frac{e^{-16\beta J} + e^{16 \beta J} + 2}{e^{16\beta J} + e^{-16\beta J} + 16e^{-8\beta J} + 16e^{8 \beta J} + 66 }\right]
		\end{flalign*}


	\subsection{Susceptibility}
		The variance of the absolute value of the mean magnetization is given by:
		\begin{flalign*}
			\sigma_{M}^2 &= \left<|M(T)^2|\right> - \left<|M(T)|\right>^2\\
			\sigma^2_M &= 16 \frac{e^{8\beta J} + 1}{e^{8 \beta J} + e^{-8 \beta J} + 8} - 16\frac{e^{16\beta J} + 4e^{8\beta J} + 4}{e^{-16\beta J} + e^{-16\beta J} + 16e^{-8\beta J} + 16e^{8 \beta J} + 66}
		\end{flalign*}

		We then calculate the susceptibility $\chi$, which is given by $\chi = \sigma_M^2/k_BT$.

		\begin{flalign*}
			\chi = \frac{16}{k_BT} \left[{\frac{e^{8\beta J} + 1}{e^{8 \beta J} + e^{-8 \beta J} + 8} -\frac{e^{16\beta J} + 4e^{8\beta J} + 4}{e^{-16\beta J} + e^{-16\beta J} + 16e^{-8\beta J} + 16e^{8 \beta J} + 66}}\right]
		\end{flalign*}

\section{Bibliography}


\end{document}
