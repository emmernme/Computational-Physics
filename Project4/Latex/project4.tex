\documentclass{article}
\usepackage{graphicx}
\usepackage[utf8]{inputenc}
\usepackage[fleqn]{amsmath}
\usepackage{titling}
\usepackage{graphicx,wrapfig,lipsum}
\usepackage{amssymb}
\usepackage{listings}
\usepackage[font=small,labelsep=none]{caption}
\usepackage{array}% http://ctan.org/pkg/array

\setlength{\droptitle}{-10em}

\title{Project 4}\vspace{-3ex}
\author{Benedicte Allum Pedersen, Emil Helland Broll & Fredrik Oftedal Forr}
\date{\vspace{-5ex}}

\begin{document}
\maketitle


\section{Abstract}

\newpage

\tableofcontents{}

\newpage

\section{Introduction}
In this project we will study the Ising model in two dimensions. This is a model which is used to simulate phase transitions. The model exhibits a phase transition from a magnetic phase to a phase with zero magnetization. The temperature where this phase transition occurs is called the critical temperature, $T_C$. Above this temperature the average magnetization is zero. We study electrons in a lattice which is a binary system because each electron only can take two values, spin up or spin down. \\

The energy we get from the Ising model without an externally applied magnetic field is given by:

\begin{flalign*}
  E = -J \sum_{<kl>}^N S_kS_l
\end{flalign*}

where $s_k, s_l = \pm 1$ and represents classical spin values. N is the total number of spins and J is a coupling constant expressing the strenght of the interactions between neighboring spins. $<kl>$ indicates that we sum over the spins of the nearest neighbors. We apply periodic boundry conditions as well as the Metropolis algorithm. We also assume that we have a ferromagnetic ordering, so $J > 0$.\\

The behavior og physical quantities like the mean magnetization, the heat capacity and the susceptibility can be characterized by a power law behavior when the temperature is near $T_C$. This gives:

\begin{flalign*}
  <M(T)> &\sim (T-T_C)^{\beta},\\
  C_v(T) &\sim |T_C-T|^{\alpha},\\
  \chi(T) &\sim |T_C-T|^{\gamma},
\end{flalign*}

where $\beta = 1/8, \alpha = 0$ and $\gamma = 7/4$. \\

The correlation length is another important physical quantity which can descriped like the ones above. The correlation length, $\varepsilon$, defines the length scale at which the overall properties of a material start to differ from its bulk properties(Jensen, M.). We expect $\varepsilon$ to be of the order of the lattice spacing for $T>>T_C$. As a result of more interactions between the spins as T approaches $T_C$ the correlation length increases as we get closer to $T_C$. Then the divergent behavior of $\varepsilon$ near $T_C$ is

\begin{flalign*}
  \varepsilon(T) \sim |T_C-T|^{-\nu}.
\end{flalign*}

We will always be limited to a finite lattice and $\varepsilon$ will be proportional with the size of the lattice. The behavior of a finite lattice can then be related to the behavior of a infinitely large lattice, so the critical temperature will scale as

\begin{flalign*}
  T_C(L) - T_C(L=\infty) = aL^{-1/\nu}.
\end{flalign*}

If we set $T=T_C$ the mean magnetization, the heat capacity and the susceptibility will be

\begin{flalign*}
  <M(T)> &\sim (T-T_C)^{\beta} \rightarrow L^{-\beta/\nu},\\
  C_v(T) &\sim |T_C-T|^{\alpha} \rightarrow L^{-\alpha/\nu},\\
  \chi(T) &\sim |T_C-T|^{\gamma} \rightarrow L^{-\gamma/\nu}.
\end{flalign*}


\section{Method}

The calculations for the degenerated energies and for the magnetization for 16 different spin configurations is located in the appendix and theese calculations gives us the Table \ref{Tab: EogM} below.

	\begin{table}[h!]
		\caption{: Spinconfigurations grouped by their total energy and magnetization}
			\label{Tab: EogM}
      \centering
		\begin{tabular}{c c c c}
			$\#$ spins up & $\#$ configurations & $E^2$ & M \\
			\hline
			4 & 1 & -8J & 4 \\
			3 & 4 & 0 & 2 \\
			2 & 4 & 0 & 0 \\
			2 & 2 & +8J & 0\\
			1 & 4 & 0 & -2 \\
			0 & 1 & -8J & -4 \\
		\end{tabular}
	\end{table}

We have used the values in Table \ref{Tab: EogM} to calculate, the expectationvalues for the energy and the mean magnetization as well. These values have then been used to calculate the variance for the two physical quantities. The variance for the energi and the mean mean magnetization have respectively been used to calculate the heat capacity $C_v &= \sigma^2_E/k_BT^2$ and the susceptibility $ \chi = \sigma_M^2/k_BT$

In Table \ref{Tab: values} the values for the properties above are shown if we set $J/k_BT =\beta J = 1$.

{\renewcommand{\arraystretch}{1.5}
\begin{table}[h!]
  \caption{: Analytical values}
    \label{Tab: values}
    \centering
  \begin{tabular}{c c c}
    Property & 2x2 lattice & For a single bond \\
    \hline
    $\left<E\right>$ & -7.978586 J & -1.994647\\
    $\left<E^2\right>$ & 63.65786 $J^2$ & 15.91447 \\
    $C_v$ & 0.170865 & 0.042716\\
    $\left<|M(T)|\right>$  & 3.991970 & 0.997993 \\
    $\left<|M(T)|^2\right> $ & 15.962527 & 3.990631 \\
    $\chi$ & 0.026703 & 0.006676\\
  \end{tabular}
\end{table}


\section{Results}
For L = 2 we need 10 000 Monte Carlo cycles in order to achieve a good agreement, then the mean value of the energy is -7.9864, which is a differnce of 0.007814 from the analytical value.

\section{Discussion}

\section{Conclusion}


\section{Appendix}
\subsection{Degenerated energies and magnetization}

When calculating the degenerate energies for the case of 2x2, we start with the equation $E_i=-J\sum\limits_{\left<kl\right>}^{2}s_ks_l$\\
The case for all spin up looks like this
\begin{tabular}{c c}
  $\uparrow$ & $\uparrow$\\
  $\uparrow$ & $\uparrow$
\end{tabular}\\

And the equation will be.
\begin{flalign*}
  E_1 &= -J\sum\limits_{<kl>}^{2}s_k s_l\\
  &= -J((s_1 s_2+s_1 s_3)+(s_2 s_1+s_2s_4)+(s_3s_1+s_3s_4)+(s_4s_3+s_4s_2))\\
  &= -J((1+1) + (1+1) + (1+1) + (1+1))\\
  E_1 &= -8J
\end{flalign*}
The reason why the same interaction is included several times is because of the unit cell repeating itself to infinity in both x and y derection. Therefore the $s_1$ will interact with $s_2$ and $s_3$ inside the unit cell, and $s_2$ and $s_3$ "outside" the unit cell.

The magnetization is defined as
\begin{flalign*}
  M_i=\sum_{j=1}^{N} s_j
\end{flalign*}
which sums over all spins for a given configuration $i$. For the same spinconfiguration as above this gives:
\begin{flalign*}
  M_i=\sum_{j=1}^{4} (1 + 1 + 1 +1) = 4
\end{flalign*}

\subsection{The partition function}
When we known the values for all the degenerate energies we can calculate the value of the partian function.
\begin{flalign*}
  &z = \sum\limits_{i=1}^{2^n}e^{-\beta E_i}\\
  &\text{In out case we have $n=4$ since we have a 2x2 lattice.}\\
  &z = \sum\limits_{i=1}^{2^4}e^{-\beta E_i}\\
  &z = e^{-\beta E_1}+e^{-\beta E_2}+\hdots+e^{-\beta E_16}\\
  &z = e^{8 \beta J}+4e^{- \beta \cdot 0} + 2e^{-8 \beta J} + 4e^{-\beta \cdot 0} + 4e^{-\beta \cdot 0} + 4e^{-\beta \cdot 0} + e^{8 \beta J}\\
  &z = 2e^{8 \beta J} + 2e^{-8 \beta J} + 16
\end{flalign*}

\subsection{Expectationvalues for the energy}
This gives us the ability to calculate the expectationvalue of the energy $\left<E\right>$

\begin{flalign*}
  \left<E\right> &= \sum\limits_{i}^{2^n}\frac{E_i e^{-\beta E_i}}{z}\\
  &\text{We know we have several energyvalues which is zero. If we do not write these we get}\\
  &= \frac{-8Je^{8\beta J} +2\left(8Je^{-8 \beta J} \right) + \left(-8J\right)e^{8\beta J}} {2e^{8 \beta J} + 2e^{-8 \beta J} + 16}\\
  &= \frac{16J\left(e^{-8\beta J}- e^{8 \beta J} \right) }{2 {\left(e^{8 \beta J} + e^{-8 \beta J} + 8 \right)} }\\
  &= 8J \frac{e^{-8\beta J}- e^{8 \beta J}}{e^{8 \beta J} + e^{-8 \beta J} + 8}
\end{flalign*}

Thus
\begin{flalign*}
  \left<E\right>^2 &= \left( 8J \frac{e^{-8\beta J}- e^{8 \beta J}}{e^{8 \beta J} + e^{-8 \beta J} + 8} \right)^2\\
  &= 64J^2 \frac{\left(e^{-8\beta J} - e^{8\beta J} \right)^2 }{\left(e^{8\beta J} + e^{-8 \beta J} + 8 \right)^2 }\\
  &= 64 J^2 \frac{e^{-16\beta J} + e^{16 \beta J} + 2}{e^{16\beta J} + e^{-16\beta J} + 16e^{-8\beta J} + 16e^{8 \beta J} + 66 }
\end{flalign*}

At the same time we calculate $\left<E^2\right>$

\begin{flalign*}
  \left<E^2\right> &= \sum\limits_{i}^{2^n}\frac{E_i^2 e^{-\beta E_i}}{z}\\
  &= \frac{64J^2e^{8\beta J} + 2\left(64J^2e^{-8\beta J}\right) + 64J^2e^{8\beta J}}{2e^{8 \beta J} + 2e^{-8 \beta J} + 16}\\
  &= \frac{128J^2e^{8\beta J} + 128J^2e^{-8\beta J}}{2e^{8 \beta J} + 2e^{-8 \beta J} + 16}\\
  &= 64J^2\frac{e^{8\beta J} + e^{-8\beta J}}{e^{8 \beta J} + e^{-8 \beta J} + 8}
\end{flalign*}

\subsection{Expectationvalues for the magnetization}
The absolute value of the mean magnetization is given by
\begin{flalign*}
  \left<|M(T)|\right> = \frac{\sum \limits{_i^{2^4}} |M_i| e^{-\beta E_i}}{z}
\end{flalign*}
and can be calculated by using the values for the energy and the magnetization from Table \ref{Tab: EogM}:

\begin{flalign*}
  \left<|M(T)|\right> &= \frac{4e^{8\beta J} + 2\cdot4 + 0\cdot4 + 0\cdot2 + |-2|\cdot4 + |-4|e^{8\beta J}}{z}\\
  \left<|M(T)|\right> &= \frac{8e^{8\beta J} + 16}{2e^{8 \beta J} + 2e^{-8 \beta J} + 16} = \frac{4e^{8\beta J} + 8}{e^{8 \beta J} + e^{-8 \beta J} + 8}
\end{flalign*}

We also have that
\begin{flalign*}
  \left<|M(T)|^2\right> &=  \frac{\sum \limits{_i^{2^4}} |M_i^2| e^{-\beta E_i}}{z}\\
  \left<|M(T)|^2\right> &= \frac{32e^{8\beta J} + 32}{2e^{8 \beta J} + 2e^{-8 \beta J} + 16} = 16\frac{e^{8\beta J} + 1}{e^{8 \beta J} + e^{-8 \beta J} + 8}
\end{flalign*}

\subsection{Heat Capacity}
We the expectationvalues to calculate the variance for the energy
\begin{flalign*}
  \sigma^2_E &= \left<E^2\right> - \left<E\right>^2\\
  &= 64J^2\frac{e^{8\beta J} + e^{-8\beta J}}{e^{8 \beta J} + e^{-8 \beta J} + 8} - 64 J^2 \frac{e^{-16\beta J} + e^{16 \beta J} + 2}{e^{16\beta J} + e^{-16\beta J} + 16e^{-8\beta J} + 16e^{8 \beta J} + 66 }
\end{flalign*}

When this is known we can also calculate the heat capasitace.
\begin{flalign*}
  C_v &= \frac{\sigma^2_E}{k_BT^2}\\
  &= \frac{64J^2}{k_BT^2}\left[\frac{e^{8\beta J} + e^{-8\beta J}}{e^{8 \beta J} + e^{-8 \beta J} + 8} -  \frac{e^{-16\beta J} + e^{16 \beta J} + 2}{e^{16\beta J} + e^{-16\beta J} + 16e^{-8\beta J} + 16e^{8 \beta J} + 66 }\right]
\end{flalign*}


\subsection{Susceptibility}
The variance for the absolute value of the mean magnetization is given by:
\begin{flalign*}
  \sigma_{M}^2 &= \left<|M(T)^2|\right> - \left<|M(T)|\right>^2\\
  \sigma^2_M &= 16 \frac{e^{8\beta J} + 1}{e^{8 \beta J} + e^{-8 \beta J} + 8} - 16\frac{e^{16\beta J} + 4e^{8\beta J} + 4}{e^{-16\beta J} + e^{-16\beta J} + 16e^{-8\beta J} + 16e^{8 \beta J} + 66}
\end{flalign*}

We then calculate the susceptibility $\chi$ which is given by $\chi = \sigma_M^2/k_BT$, so

\begin{flalign*}
  \chi = \frac{16}{k_BT} \left[{\frac{e^{8\beta J} + 1}{e^{8 \beta J} + e^{-8 \beta J} + 8} -\frac{e^{16\beta J} + 4e^{8\beta J} + 4}{e^{-16\beta J} + e^{-16\beta J} + 16e^{-8\beta J} + 16e^{8 \beta J} + 66}}\right]
\end{flalign*}

\section{Bibliography}


\end{document}
